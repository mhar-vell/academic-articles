\section{Exemplos de comandos}

\subsection{Margens}

A norma ABNT NBR 6022:2018 não estabelece uma margem específica a ser utilizada
no artigo científico. Dessa maneira, caso deseje alterar as margens, utilize os
comandos abaixo:

\begin{verbatim}
   \setlrmarginsandblock{3cm}{3cm}{*}
   \setulmarginsandblock{3cm}{3cm}{*}
   \checkandfixthelayout
\end{verbatim}

\subsection{Duas colunas}

É comum que artigos científicos sejam escritos em duas colunas. Para isso,
adicione a opção \texttt{twocolumn} à classe do documento, como no exemplo:

\begin{verbatim}
   \documentclass[article,11pt,oneside,a4paper,twocolumn]{abntex2}
\end{verbatim}

É possível indicar pontos do texto que se deseja manter em apenas uma coluna,
geralmente o título e os resumos. Os resumos em única coluna em documentos com
a opção \texttt{twocolumn} devem ser escritos no ambiente
\texttt{resumoumacoluna}:

\begin{verbatim}
   \twocolumn[              % INICIO DE ARTIGO EM DUAS COLUNAS

     \maketitle             % pagina de titulo

     \renewcommand{\resumoname}{Nome do resumo}
     \begin{resumoumacoluna}
        Texto do resumo.
      
        \vspace{\onelineskip}
 
        \noindent
        \textbf{Palavras-chave}: latex. abntex. editoração de texto.
     \end{resumoumacoluna}
   
   ]                        % FIM DE ARTIGO EM DUAS COLUNAS
\end{verbatim}

\subsection{Recuo do ambiente \texttt{citacao}}

Na produção de artigos (opção \texttt{article}), pode ser útil alterar o recuo
do ambiente \texttt{citacao}. Nesse caso, utilize o comando:

\begin{verbatim}
   \setlength{\ABNTEXcitacaorecuo}{1.8cm}
\end{verbatim}

Quando um documento é produzido com a opção \texttt{twocolumn}, a classe
\textsf{abntex2} automaticamente altera o recuo padrão de 4 cm, definido pela
ABNT NBR 10520:2002 seção 5.3, para 1.8 cm.

\section{Cabeçalhos e rodapés customizados}

Diferentes estilos de cabeçalhos e rodapés podem ser criados usando os
recursos padrões do \textsf{memoir}.

Um estilo próprio de cabeçalhos e rodapés pode ser diferente para páginas pares
e ímpares. Observe que a diferenciação entre páginas pares e ímpares só é
utilizada se a opção \texttt{twoside} da classe \textsf{abntex2} for utilizado.
Caso contrário, apenas o cabeçalho padrão da página par (\emph{even}) é usado.

Veja o exemplo abaixo cria um estilo chamado \texttt{meuestilo}. O código deve
ser inserido no preâmbulo do documento.

\begin{verbatim}
%%criar um novo estilo de cabeçalhos e rodapés
\makepagestyle{meuestilo}
  %%cabeçalhos
  \makeevenhead{meuestilo} %%pagina par
     {topo par à esquerda}
     {centro \thepage}
     {direita}
  \makeoddhead{meuestilo} %%pagina ímpar ou com oneside
     {topo ímpar/oneside à esquerda}
     {centro\thepage}
     {direita}
  \makeheadrule{meuestilo}{\textwidth}{\normalrulethickness} %linha
  %% rodapé
  \makeevenfoot{meuestilo}
     {rodapé par à esquerda} %%pagina par
     {centro \thepage}
     {direita} 
  \makeoddfoot{meuestilo} %%pagina ímpar ou com oneside
     {rodapé ímpar/onside à esquerda}
     {centro \thepage}
     {direita}
\end{verbatim}

Para usar o estilo criado, use o comando abaixo imediatamente após um dos
comandos de divisão do documento. Por exemplo:

\begin{verbatim}
   \begin{document}
     %%usar o estilo criado na primeira página do artigo:
     \pretextual
     \pagestyle{meuestilo}
     
     \maketitle
     ...
     
     %%usar o estilo criado nas páginas textuais
     \textual
     \pagestyle{meuestilo}
     
     \chapter{Novo capítulo}
     ...
   \end{document}  
\end{verbatim}

Outras informações sobre cabeçalhos e rodapés estão disponíveis na seção 7.3 do
manual do \textsf{memoir} \cite{memoir}.

\section{Imagens}
%-----------------------------especificar o tamanho------------------------------------------------------------------------------------------
\begin{figure}[htbp]
    \centerline{
        \includegraphics*[scale=1.5]{imagens/fig1.png} 
        }
\caption{Example of a figure caption.}
\label{fig}
\end{figure}
% %-----------------------------imagem emoldurada--------------------------------------------------------------------------------------------
% \begin{figure}[htbp]
%    \centerline{
%        \fbox{\includegraphics*[scale=1.5]{imagens/fig1.png}} 
%        }
% \caption{Example of a figure caption.}
% \label{fig}
% \end{figure}
% %-----------------------------especificar o altura e largura-------------------------------------------------------------------------------
% \begin{figure}[htbp]
%     \centerline{
%         \includegraphics*[width=2cm, height=3cm]{imagens/fig1.png} 
%         }
% \caption{Example of a figure caption.}
% \label{fig}
% \end{figure}
% %-----------------------------setar a mesma largura do texto-------------------------------------------------------------------------------
% \begin{figure}[htbp]
%     \centerline{
%         \includegraphics*[width=\textwidth]{imagens/fig1.png} 
%         }
% \caption{Example of a figure caption.}
% \label{fig}
% \end{figure}
% %-----------------------------imagem inscrita em outra-------------------------------------------------------------------------------------
% \begin{figure}[ht]
%     \includegraphics[scale=1]{example-image-1x1}
%     \centering
%     \llap{\shortstack{%
%             \includegraphics[scale=1.5]{imagens/fig1.png}\\
%             \rule{0ex}{1.3in}%polegadas eixo Y
%           }
%       \rule{0.2in}{0ex}}%polegadas eixo X
%     \caption{This is my embedded figure}
%     \end{figure}
% %-----------------------------4 imagens inscritas em outra---------------------------------------------------------------------------------
% \begin{figure}[ht]
%    \includegraphics[scale=1]{example-image-1x1}
%    \centering
%    %superior esquerda
%    \llap{\shortstack{
%            \includegraphics[scale=1.5]{imagens/fig1.png}\\
%            \rule{0ex}{1.35in}%polegadas eixo Y
%          }
%      \rule{1.4in}{0ex}}%polegadas eixo X
%    %inferior esquerda
%    \llap{\shortstack{
%            \includegraphics[scale=1.5]{imagens/fig1.png}\\
%            \rule{0ex}{0.15in}%polegadas eixo Y
%         }
%      \rule{1.4in}{0ex}}%polegadas eixo X
%    %superior direita  
%    \llap{\shortstack{
%             \includegraphics[scale=1.5]{imagens/fig1.png}\\
%             \rule{0ex}{1.35in}%polegadas eixo Y
%          }
%       \rule{0.2in}{0ex}}%polegadas eixo X
%    %inferior direita
%    \llap{\shortstack{
%             \includegraphics[scale=1.5]{imagens/fig1.png}\\
%             \rule{0ex}{0.15in}%polegadas eixo Y
%          }
%       \rule{0.2in}{0ex}}%polegadas eixo X
%    \caption{This is my embedded figure}
%    \end{figure}
% %-----------------------------sombra na imagem---------------------------------------------------------------------------------------------
% \begin{figure}[htbp]
%    \centerline{
%        \shadowbox{\includegraphics*[scale=1.5]{imagens/fig1.png}} 
%        }
% \caption{Example of a figure caption.}
% \label{fig}
% \end{figure}
%-----------------------------Quebra de texto em torno da figura-----------------------------------------------------------------------------
% Praesent in sapien. Lorem ipsum dolor sit amet, consectetuer adipiscing elit. Duis fringilla tristique neque. Sed interdum libero ut metus. Pellentesque placerat.

% \begin{wrapfigure}{l}{0.25\textwidth}
% \includegraphics[width=0.9\linewidth]{imagens/fig1.png} 
% \caption{Caption1}
% \label{fig:wrapfig}
% \end{wrapfigure}

% Praesent in sapien. Lorem ipsum dolor sit amet, consectetuer adipiscing elit. Duis fringilla tristique neque. Sed interdum libero ut metus. Pellentesque placerat. Nam rutrum augue a leo. Morbi sed elit sit amet ante lobortis sollicitudin.

% Praesent in sapien. Lorem ipsum dolor sit amet, consectetuer adipiscing elit. Duis fringilla tristique neque. Sed interdum libero ut metus. Pellentesque placerat. Nam rutrum augue a leo. Morbi sed elit sit amet ante lobortis sollicitudin.Praesent in sapien. Lorem ipsum dolor sit amet, consectetuer adipiscing elit. Duis fringilla tristique neque. Sed interdum libero ut metus. Pellentesque placerat. Nam rutrum augue a leo. Morbi sed elit sit amet ante lobortis sollicitudin. Pellentesque placerat. Nam rutrum augue a leo. Morbi sed elit sit amet ante lobortis sollicitudin.mauri.Praesent in sapien. Lorem ipsum dolor sit amet, consectetuer adipiscing elit. Duis fringilla tristique neque. Maecenas dignissim aliquam pellentesque. 

%-----------------------------posição a direita--------------------------------------------------------------------------------------------
% Lorem ipsum dolor sit amet, consectetuer adipiscing elit. Etiam lobortis facilisis sem. Nullam nec mi et neque pharetra sollicitudin.
% \begin{figure}[htbp]
% \includegraphics[width=0.3\textwidth, right]{imagens/fig1.png}
% \end{figure}

% Praesent imperdiet mi nec
% ante. Donec ullamcorper, felis non sodales commodo, lectus velit ultrices augue,
% a dignissim nibh lectus placerat pede. Vivamus nunc nunc, molestie ut, ultricies
% vel, semper in, velit. Ut porttitor.

%-----------------------------posição a esquerda--------------------------------------------------------------------------------------------
% Lorem ipsum dolor sit amet, consectetuer adipiscing elit. Etiam lobortis facilisis sem. Nullam nec mi et neque pharetra sollicitudin.
% \begin{figure}[htbp]
% \includegraphics[width=0.3\textwidth]{imagens/fig1.png}
% \end{figure}

% Praesent imperdiet mi nec
% ante. Donec ullamcorper, felis non sodales commodo, lectus velit ultrices augue,
% a dignissim nibh lectus placerat pede. Vivamus nunc nunc, molestie ut, ultricies
% vel, semper in, velit. Ut porttitor.
%-----------------------------2 imagens em 1 figura----------------------------------------------------------------------------------------
% \begin{figure}[h]
   
%    \begin{subfigure}{0.5\textwidth}
%    \includegraphics[width=0.9\linewidth, height=6cm]{imagens/fig1.png} 
%    \caption{Caption1}
%    \label{fig:subim1}
%    \end{subfigure}
%    \begin{subfigure}{0.5\textwidth}
%    \includegraphics[width=0.9\linewidth, height=6cm]{imagens/fig1.png}
%    \caption{Caption 2}
%    \label{fig:subim2}
%    \end{subfigure}
   
%    \caption{Caption for this figure with two images}
%    \label{fig:image2}
%    \end{figure}
   